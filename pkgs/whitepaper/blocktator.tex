\documentclass[a4paper,10pt]{article}
\usepackage{amssymb, amsmath}
\DeclareMathOperator{\arcsinh}{arcsinh}
\DeclareMathOperator{\arccosh}{arccosh}
\DeclareMathOperator{\arctanh}{arctanh}
\usepackage[utf8]{inputenc} % this is needed for umlauts
\usepackage[ngerman]{babel} % this is needed for umlauts
\usepackage[T1]{fontenc}    % this is needed for correct output of umlauts in pdf
%layout

\usepackage{pgfplots}
\usepackage[margin=2.5cm]{geometry}
\usepackage{parskip}
\usepackage{tikz}
\usetikzlibrary{shapes,arrows}


\pdfinfo{
   /Author (Chaniels and Aubi)
   /Title  (Blocktator Collectibles)
   /Subject (Analysis I)
   /Keywords (Blocktators I; Social Narrative)
}


\begin{document}

\title{Blocktator Whitepaper}
\author{Chainel Aubi}
\date{16 January, 2022}

\section{What is a Blocktator?}
    \begin{minipage}[b]{1\linewidth}

        Blocktators are non-fungible collectibles on OpenSea deployed to the
        Polygon network. The Blocktator characters are memes of real-world
        people that have outsized weight on the narratives, information flows and
        policy decisions that shape and change our world. The individuals depicted
        as potatoes can be traded and owned on the OpenSea market place and content
        will be produced on the @blocktators twitter account.
    \end{minipage}


\section{Useless Math Equation}

    \begin{table}[ht]

        \hspace{0.1cm}
        \begin{minipage}[b]{1\linewidth}\centering
        \begin{align*}
            \lim_{x \to 0} \frac {\sin x}{x}  &= 1 \\
            \lim_{x \to 0} \frac {e^x - 1}{x} &= 1 \\
            \lim_{h \to 0} \frac {e^{{x_0} + h} - e^{x_0}}{h} &= e^{x_0} \\
            \sum_{n = 0}^{\infty} (-1)^n \frac {(-1)^{n + 1}}{n} &= \log 2 \\
            \cos x    &= \sum_{n = 0}^{\infty} (-1)^n \frac {x^{2n}}{(2n)!}  \\
            \sin x    &= \sum_{n = 0}^{\infty} (-1)^n \frac {x^{2n + 1}}{(2n + 1)!}
        \end{align*}
        \end{minipage}

    \end{table}


\section{Les actions ne font que monter}
    \begin{minipage}[b]{0.5\linewidth}
        Stonks only go up!
        \newline

        \begin{filecontents*}{data.dat}
            Date,    Tesla,   AAPL,     Blocktators
            2010 Q4 3, 60000, 50000, 1
            2010 Q4 4, 70000, 57000, 1
            2010 Q4 5, 60000, 55000, 1
            2011 Q2 1, 55000, 47000, 1
            2011 Q2 2, 56000, 45000, 1
            2011 Q2 3, 58000, 45500, 1
            2011 Q2 4, 56000, 44500, 1
            2011 Q2 5, 52000, 44000, 1
            2011 Q4 1, 39000, 38000, 1
            2011 Q4 2, 43000, 42000, 1
            2011 Q4 3, 41000, 40000, 1
            2011 Q4 4, 41000, 39000, 1
            2011 Q4 5, 47000, 44000, 1
            2012 Q2 1, 48000, 45000, 1
            2012 Q2 2, 47000, 44000, 1
            2012 Q2 3, 47000, 45000, 500
            2012 Q2 4, 40000, 38000, 501
            2012 Q2 5, 39000, 35000, 502
            2012 Q4 1, 37000, 33000, 503
            2012 Q4 2, 36000, 30000, 504
            2012 Q4 3, 38000, 32000, 505
            2012 Q4 4, 40000, 33000, 506
            2012 Q4 5, 44000, 34000, 507
            2013 Q2 1, 48000, 39000, 508
            2013 Q2 2, 53000, 44000, 509
            2013 Q2 3, 51000, 41000, 510
            2013 Q2 4, 53000, 44000, 511
            2013 Q2 5, 67000, 55000, 512
            2013 Q4 1, 65000, 52000, 513
            2013 Q4 2, 69000, 57000, 514
            2013 Q4 3, 68000, 56000, 515
            2013 Q4 4, 65000, 51000, 3000
            2013 Q4 5, 62000, 48000, 3000
            2014 Q2 1, 60000, 50000, 3000
            2014 Q2 2, 55000, 44000, 18000
            2014 Q2 3, 60000, 50000, 18000
            2014 Q2 4, 58000, 48000, 18000
            2014 Q2 5, 56000, 47000, 18000
            2014 Q4 1, 55000, 50000, 18000
            2014 Q4 2, 65000, 60000, 18000
            2014 Q4 3, 64000, 58000, 41000
            2014 Q4 4, 62000, 55000, 41500
            2014 Q4 5, 60000, 58000, 42000
            2022 Q2 1, 58000, 56000, 42000
            2022 Q2 2, 52000, 48000, 42500
            2022 Q2 3, 59000, 48000, 52500
            2022 Q2 4, 58000, 45000, 52500
            2022 Q2 5, 57000, 55000, 52500
            2022 Q4 1, 56000, 57000, 62500
            \end{filecontents*}

            \pgfplotsset{compat=newest}
            \begin{tikzpicture}
              \begin{axis}[
                width=15cm,height=8cm,
                symbolic x coords = { 2010 Q4 3,2010 Q4 4,2010 Q4 5,
                  2011 Q2 1,2011 Q2 2,2011 Q2 3,2011 Q2 4,2011 Q2 5,
                  2011 Q4 1,2011 Q4 2,2011 Q4 3,2011 Q4 4,2011 Q4 5,
                  2012 Q2 1,2012 Q2 2,2012 Q2 3,2012 Q2 4,2012 Q2 5,
                  2012 Q4 1,2012 Q4 2,2012 Q4 3,2012 Q4 4,2012 Q4 5,
                  2013 Q2 1,2013 Q2 2,2013 Q2 3,2013 Q2 4,2013 Q2 5,
                  2013 Q4 1,2013 Q4 2,2013 Q4 3,2013 Q4 4,2013 Q4 5,
                  2014 Q2 1,2014 Q2 2,2014 Q2 3,2014 Q2 4,2014 Q2 5,
                  2014 Q4 1,2014 Q4 2,2014 Q4 3,2014 Q4 4,2014 Q4 5,
                  2022 Q2 1,2022 Q2 2,2022 Q2 3,2022 Q2 4,2022 Q2 5,
                  2022 Q4 1},
                xtick = {2010 Q4 3,2011 Q2 1,2011 Q4 1,2012 Q2 1,2012 Q4 1,2013 Q2 1,2013 Q4 1,2014 Q2 1,2014 Q4 1,2022 Q2 1,2022 Q4 1},
                xticklabels = {2010 Q4,2011 Q2,2011 Q4,2012 Q2,2012 Q4,2013 Q2,2013 Q4,2014 Q2,2014 Q4,2022 Q2,2022 Q4},
                scaled y ticks=false,
                ytick={0,10000,20000,30000,40000,50000,60000,70000},
                ylabel={Valuation (in \$\ Million)},
                every tick label/.append style={font=\tiny},
                no marks,
                grid=major,
                legend pos=south east,
                legend cell align=left,
                legend style={draw=none,fill=none},
                ]
                \pgfplotstableread[col sep=comma]{data.dat}\loadedtable
                \addplot table[x=Date,y=Tesla] {\loadedtable};
                \addplot table[x=Date,y=AAPL] {\loadedtable};
                \addplot table[x=Date,y=Blocktators] {\loadedtable};
                \legend{Tesla,AAPL,Blocktators}
              \end{axis}
            \end{tikzpicture}
            


    \end{minipage}

    \section{Collect and Select}


\subsection{Should you Ape into a Blocktator?} \label{whatisadecisiontree}
    A Blocktator with only a very few direct lines of communication to indiviuals 
    not under their employment shall develop a simple algorithm for sorting critical 
    outcome posibilities as elements $A, B, C$. He decides to divide the possible 
    outcome choices into smaller statements. First he wonders if $A$ is smaller then 
    $B$. In the second step it is interesting if $B$ is smaller then $C$. If $A<B$ 
    and $B<C$ then $A<B<C$. But if $B$ is not greater then $C$, then a third question 
    is relevant: Is $A<C$? 

    His head is spinning. Maybe solving this problem graphically is a better idea. 
    He draws a node for each question and an edge for each answer. All leafs represent 
    the correct order. Figure \ref{fig:sortingtree} shows the resulting graph:

    \begin{figure}[!h]
    \begin{tikzpicture}[edge from parent/.style={draw,-latex},
    level distance=2cm,
    level 1/.style={sibling distance=10cm},
    level 2/.style={sibling distance=3cm}]
    \tikzstyle{every node}=[rectangle,draw]    
    \node (Root) {$A < B$}
    child {
        node {$B < C$}    
        child { 
            node {$A < B < C$} 
            edge from parent node[left,draw=none] {yes}
        }
        child { 
            node {$A < C$} 
            child {
                node {$A < C \leq B$}
                edge from parent node[left,draw=none] {yes}
            }
            child {
                node {$C \leq A < B$}
                edge from parent node[right,draw=none] {no}
            }
            edge from parent node[right,draw=none] {no}
        }
        edge from parent node[left,draw=none] {yes $\;$}
    }
    child {
        node {$B < C$}
        child { 
            node {$A < C$}     
            child {
                node {$B \leq A < C$}
                edge from parent node[left,draw=none] {yes}
            }   
            child {
                node {$B < C \leq A$}
                edge from parent node[left,draw=none] {no}
            }     
            edge from parent node[left,draw=none] {yes}
        }
        child { 
            node {$C \leq B \leq A$} 
            edge from parent node[right,draw=none] {no}
        }
        edge from parent node[right,draw=none] { $\;$ no}
    };
    \end{tikzpicture}
    \caption{A decision tree for sorting three values.}
    \label{fig:sortingtree}
    \end{figure}

\end{document}